\documentclass[11pt]{article}
\usepackage[bahasa]{babel}
\begin{document}

\tableofcontents
\title{Judul dalam \LaTeX\ ya}
\author{Sandy Herho}
\date{\today}
\maketitle

Cara menuliskan teks \textit{miring} di LaTeX.

Cara menuliskan teks \textbf{tebal} di LaTeX.

Cara menuliskan huruf \textsc{kapital} di LaTeX.

Cara menuliskan font \texttt{mesin ketik} di LaTeX.

Silakan untuk melihat profil saya di \texttt{http://orcid.org/0000-0001-8330-2095}.

Nama saya Sandy Herho.

Nama saya \begin{large}Sandy Herho\end{large}.

Nama saya \begin{Large}Sandy Herho\end{Large}.

Nama saya \begin{huge}Sandy Herho\end{huge}.

Nama saya \begin{small}Sandy Herho\end{small}.

Nama saya\begin{tiny} Sandy Herho\end{tiny}.

\begin{center}
Tengah
\end{center}

\begin{flushleft}
Rata kiri
\end{flushleft}

\begin{flushright}
rata kanan
\end{flushright}

\section{Python}
	\subsection{Dasar}
	Python merupakan bahasa pemrograman universal yang dapat memecahkan permasalahan $\sqrt[4]{\frac{1}{1+\sqrt{\frac{\phi^2}{5}}}}$ lho.
	\subsection{NumPy}
	\subsection{SciPy}
	\subsection{Matplotlib}
	\subsection{Scikit}
	\subsection{Pandas}
	\subsection{TensorFlow} \hfill\section{Geologi}
	\subsection{Petrologi}
		
	\subsection{Sedimentologi}
	\subsection{Stratigrafi}
	\subsection{Geomorfologi}
	\subsection{Geologi Struktur}
	\subsection{Paleontologi}

\end{document}